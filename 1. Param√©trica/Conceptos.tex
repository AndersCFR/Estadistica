\documentclass{article}
\usepackage[utf8]{inputenc} %para caracteres en español%
\usepackage[spanish]{babel} %para correcciones en español%
\setlength{\parskip}{5px} %espaciado de párrafos%
\author{Anderson Cárdenas}
\title{Conceptos estadísticos}
\begin{document}
    \maketitle
    \section{Definiciones}

    \begin{itemize}
        \item \textbf{Población:} Conjunto de todos los individuos de estudio.
        \item \textbf{Muestra:} Subconjunto de la población.
        \item \textbf{Muestra aleatoria simple (MAS):} La muestra debe ser representativa, es decir que 
        representa las características de la población. Implica que todos los elementos tienen la misma 
        probabilidad de ser escogidas para formar parte de la muestra. Se definen dos casos: 
        \item \textbf{Muestreo con reemplazo:} La muestra contiene elementos repetidos.
        \item \textbf{Muestreo sin reemplazo:} La muestra no contiene elementos repetidos.
        \item \textbf{Variable aleatoria(VA):} Es una función que asigna un resultado numérico como 
        resultado de un experimento aleatorio.
        \item \textbf{VA discreta:} El rango es un conjunto finito o infinito numerable.
        \item \textbf{VA continua:} El rango es un conjunto finito o infinito no numerable.        

    \end{itemize}

    \section{Estadística Descriptiva}
\end{document}